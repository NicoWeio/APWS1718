\section{Durchführung}
\label{sec:Durchführung}
Für diesen Experiment sollen die Strömungsgeschwindigkeiten sowie die Strömungsprofile näher untersucht werden. Dabei steht ein Schlauchsystem, also drei Strömungsröhre mit verschiedenen Durchmessern und eine rote Sonde, die eine Frequenz von $\SI{2}{\mega\hertz}$ besitzt, zur Verfügung. Um den Doppler-Effekt zu zeigen, wird es nicht mit dem Ultraschallkopf nicht an dem Schlauchsystem gehalten, sondern es werden drei Doppler-Prismen mit drei verschiedenen Prismawinkel verwendet.

Die Prismen besitzen die Winkel $15 ^\circ$, $30 ^\circ$, und $60 ^\circ$. Die Bestimmung des Dopplerwinkels wird mit Hilfe des Brechungsgesetz und mit Hilfe von verschiedenen Schallgeschwindigkeiten in Wasser und im Prisma berechnet:

\begin{equation}
\alpha = 90^\circ - \arcsin\left(\sin\theta\cdot\frac{c_\text{L}}{c_\text{P}}\right).
\end{equation}

Für den ersten Teil wird die Frequenzverschiebung $\Delta\nu$ für $5$ verschiedene Flussgeschwindigkeiten mit der $\SI{2}{\mega\hertz}$ Sonde für alle drei Winkel notiert. Mittels Ultraschallgel werden die Dopplerprismen an die Strömungsröhre gebracht. 

Im zweiten Teil wird das Strömungsprofil an dem mittleren Schlauch näher untersucht. Dabei ist es hier nur den $15 ^\circ$-Winkel notwendig. Die Meßtiefe wird variiert und wird zuerst bei $12$ eingestellt. Auch die Pumpleistung wird auf maximal $\SI{70}{\percent}$ und danach auf $\SI{45}{\percent}$ gestellt. Es werden jeweils die Werte für die Streuintensität und Strömungsgeschwindigkeit notiert. 