\section{Diskussion}
Schon bei der Berechnung der Apparatekonstante fällt auf, dass die benötigte Viskosität von Wasser bei Raumtemperatur recht stark von ihrem Literaturwert von $1 mPa \cdot s$ abweicht. Diese Abweichung beträgt rund 23\%.
Da die Fallzeit und die Dichte der kleinen Kugel jeweils geringe Fehlerwerte hatten, könnte man vermuten, dass die gegebene Apparatekonstante der kleinen Kugel zu ungenau bestimmt wurde.
Dennoch weist die Apparatekonstante $K_{\text{gr}}$, welche unter anderem mit dieser Viskosität berechnet wurde, einen geringen Fehler auf. Daher kann man davon ausgehen, dass dieser Fehler sich nicht allzu stark auf diese
weiteren Rechnungen auswirkte. 
Weiterhin liegen die berechneten Reynoldszahlen sehr deutlich unter dem kritischen Wert von 1160, woraus zu schließen ist, dass die Strömung im Rohr laminar ist und Wirbel nicht die Durchführung oder die Ergebnisse beeinflusst 
haben.

Zuletzt ist in Abbildung \eqref{fig:bildandrade} zu erkennen, dass die meisten Messwerte recht nah an der Ausgleichsgeraden liegen, sodass diese, als auch die daraus bestimmten Konstanten der Andradeschen Gleichung als 
relativ genau betrachtet werden können. 