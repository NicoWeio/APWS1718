\section{Versuchsdurchführung}
\subsection{Bestimmung der Apparatekonstanten}
Um später die Trägheitsmomente unterschiedlicher Körper berechnen zu können, müssen die Apparatekonstanten D und $I_D$ bekannt sein.
Zur Bestimmung der Winkelrichtgröße D wird eine annähernd masselose Stange auf der Drillachse befestigt. An dieser wird eine Federwaage
in einem festen Abstand a zur Drehachse eingehängt und dann in mehreren Messungen in unterschiedlichen Winkel ausgelenkt.

Für die Ermittlung von $I_D$ nutzt man zwei identische Gewichte, die an jeweils einem Ende des Stabes befestigt werden.
Der Stab mit den Gewichten wird um einen bestimmten Winkel ausgelenkt und losgelassen, sodass es in Schwingung gerät.
Es wird die Periodendauer T einer Schwingung mithilfe einer Stopuhr gemessen, wobei bei jeder Messung der Abstand der Gewichte zur
Drehachse variiert wird.

\subsection{Bestimmung des Trägheitsmoments zweier Körper}
Statt der Stange wird nun ein Körper auf der Drillachse befestigt und mehrfach um einen festen Winkel ausgelenkt. Jedes Mal wird 
die Periodendauer T gemessen und notiert. Im Anschluss wird der Körper durch einen anderen ausgetauscht, mit dem auf dieselbe 
Weise verfahren wird wie zuvor.

\subsection{Bestimmung des Trägheitsmoments einer Holzpuppe}
Eine Holzpuppe wird auf der Drillachse befestigt und eine bestimmte Körperhaltung gebracht. So wie bei den beiden Körper zuvor
wird die Holzpuppe nun in Schwingung gebracht, um ihre Periodendauer zu messen. Jenach verändert man die Körperhaltung der Puppe
und wiederholt den Versuch bei gleichbleibenden Bedingungen.
