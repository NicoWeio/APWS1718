\newpage 
\section{Diskussion}
Ziel dieses Versuches war es, das magnetische Moment eines Permanentmagneten auf drei verschiedene Weisen zu berechnen.
In jeder der Messmethoden gab es Fehlerquellen, die berücksichtigt werden müssen.
Zunächst die Gravitationsmethode. Hier könnten Fehler beim Vermessen der Billiardkugel mitsamt des Stiels auftreten sein. 
Zudem ist es möglich,dass die Abstände r von der Masse bis zur Kugel nur ungenau bestimmmt wurden und es ein Intervall gab, 
in der die Masse bei einem bestimmten Magnetfeld noch stabil war. Diese ungenauen Abstände würden sich damit direkt auf die
Berechnung von $\mu_{\text{Dipol}}$ auswirken. 

Bei der Methode über die Schwingungsdauer könnte es sein, dass die Auslenkungen nicht bei jeder Messung gleich groß waren und die 
Durchführungen somit nicht einheitlich, da Periodendauern dadurch vergrößert oder verkleinert wurden. Bei kleinen Periodendauern könnten
sich außerdem Fehler beim Stoppen der Zeit ergeben haben. 

Zuletzt gab es auch bei der Präzessionsmethode einige Fehlerquellen. Zunächst musste die Kugel in Rotation und in einen stabilen Zustand
versetzt werden. Wenn dies gelungen war, präzidierte die Kugel häufig schon, bevor der Strom richtig eingestellt wurde. Außerdem kam
es auch hier, wie in der Schwingungsmethode, vor, dass die Kugel nicht immer gleich stark ausgelenkt wurde. Weiterhin ergaben sich 
sicherlich auch Fehler beim Messen der Periodendauer, da es nicht immer genau ersichtlich war, wann eine Periode erreicht wurde. 

Ferner spielte bei allen Messmethoden eine Rolle wie lange der Strom eingeschaltet war, weil beim Erwärmen des Spulendrahtes dessen
Widerstand ansteigt. Daraus würde folgen, dass das Magnetfeld schwächer wird, was sich auf die gesamte Berechnung der magnetischen
Momente der verschiedenen Messmethoden auswirkt. 

Die Ergebnisse dieses Versuchs sind die folgenden magnetischen Momente:

\begin{equation*}
\begin{aligned}
\mu_{\text{Dipol, Grav}} &= (0.0953 \pm 0.0115) \symup{\frac{J}{T}} \\
\mu_{\text{Dipol, Schwing}} &= (0.1028 \pm 2.4674) \symup{\frac{J}{T}} \\
\mu_{\text{Dipol, Präz}} &= (0.0887 \pm 0.0054) \symup{\frac{J}{T}} 
\end{aligned}
\end{equation*}

Diese Werte weichen mindestens mit $6.93\%$ und maximal mit $15.9\%$ voneinander ab. Es ist zu erkennen, dass das magnetische Moment,
das über die Schwingungsmethode bestimmt wurde, den größten Fehlerwert aufweist. Dagegen hat die Präzessionsmethode den kleinsten.
Auffällig ist hierbei, dass die Präzessionsmethode eigentlich die meisten Fehlerquellen hat, aber dennoch die Methode ist, bei 
der das magnetische Moment den kleinsten Fehlerwert hat.
