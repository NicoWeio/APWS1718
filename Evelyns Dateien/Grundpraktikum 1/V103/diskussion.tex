\pagebreak

\section{Diskussion}
Für diesen Versuch wurden zwei Kupferstäbe, rund und eckig, gewählt und untersucht. Die Literaturwerte \cite{wiki} des Elastizitätsmoduls für Kupfer liegen in einem Bereich 
zwischen $1{,}3$ und $1{,}8 \cdot 10^9 \symup{\frac{N}{m^2}}$. Die experimentell bestimmten Werte sind
\begin{equation*}
\begin{aligned}
E_{\text{e}} &= (78{,}8 \pm 1{,}6)\cdot 10^9 \symup{\frac{N}{m^2}} \\
E_{\text{r}} &= (4{,}77 \pm 0{,}03)\cdot 10^9 \, \symup{\frac{N}{m^2}}
\end{aligned}
\end{equation*}
für die einseitige Einspannung und
\begin{equation*}
\begin{aligned}
E_{\text{br}} &= (7{,}40 \pm 0{,}18)\cdot 10^9 \symup{\frac{N}{m^2}} \\
E_{\text{bl}} &= (6{,}22 \pm 0{,}17)\cdot 10^9 \symup{\frac{N}{m^2}}
\end{aligned}
\end{equation*}
für die beidseitige Auflage. Es fällt auf, dass vor allem das Elastizitätsmodul des eckigen Stabes bei einseitiger Einspannung sehr stark von dem Literaturwert abweicht und auch den größten Fehlerwert besitzt. 
Die Abweichung von $E_{\text{e}}$ zur oberen Grenze des Literaturwertes beträgt 42,8\%. Das dem Literaturwert nächste Elastizitätsmodul ist $E_{\text{r}}$ mit einer Abweichung von 1,65\%. Zudem ist $E_{\text{r}}$ der am 
genauesten bestimmte Wert, was an dem geringen Fehlerwert zu erkennen ist. Somit liefert die Messung des Elastizitätsmoduls durch einseitige Einspannung eines runden Stabs das beste Ergebnis, auch im Bezug zur 
Methode durch beidseitige Auflage, welche mit dem gleichen Stab durchgeführt wurde.

Dennoch sind bei der Durchführung einige Quellen für Messungenauigkeiten zu beobachten. Insbesondere seien hier die Messuhren genannt, deren Zeiger sich auch bei kleinen oder gar keinen Erschütterungen deutlich bewegen, 
was zu einer ungenauen Messung führt. Digitale Messuhren könnten hier bessere Ergebnisse liefern. Zudem ist es möglich, dass die Stäbe auch ohne zusätzliches Gewicht verbogen sind, was ebenfalls die Werte verfälschen könnte.
Zuletzt ist es nicht auszuschließen, dass Messwerte nicht genau genug abgelesen werden, was sowohl bei den Messuhren als auch beim Abmessen der Radien bzw. Dicken der Stäbe mit der Schieblehre der Fall sein könnte.