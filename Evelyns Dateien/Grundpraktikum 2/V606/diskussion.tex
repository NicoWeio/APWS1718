\section{Diskussion}

Die im ersten Versuchsteil bestimmte Filterkurve zeigt den erwarteten markanten Verlauf, was gut die Eigenschaften des Selektivverstärkers deutlich macht.
Der zweite Versuchsteil zur Bestimmung der Suszeptibilitäten liefert die folgenden Ergebnisse:

\begin{table}[h!tbp]
\centering
\caption{Ergebnisse der Auswertung.}
\label{tab:some_data}
\begin{tabular}{S S[table-auto-round, table-format=1.4, table-figures-uncertainty=1] S S}
\toprule
 & {$Gd_2O_3 $} & {$Dy_2O_3$} & {$Nd_2O_3$} \\
\midrule
$\chi_{\text{T}}$ & 0.0069 & 0.0126 & 0.0039 \\
$\chi_{\text{R}}$ & 0.0014 \pm 0.002  & 0.0269 \pm 0.0005  & 0.0043 \pm 0.0007  \\
\bottomrule
\end{tabular}
\end{table}
Bei $Gd_2O_3$ weicht der experimentell bestimmte Wert um $79{,}71\%$ vom Theoriewert ab, bei $Dy_2O_3$ beträgt die Abweichung $-113{,}49\%$ und bei $Nd_2O_3$ ist sie $-10{,}26\%$. Diese großen Abweichungen sprechen für keine genaue Messung,
was beispielsweise daran liegen könnte, dass die Messwerte nicht genau genug abgelesen werden können. Eine Schwierigkeit ist zudem das Einstellen der Spannung auf ein Minimum über den regelbaren Widerstand, da dieses Minimum oft nicht genau
zu erkennen ist und es für einige Widerstände Spannungen gibt, die fehlerhaft sind und somit nicht mit aufgenommen werden können. Dies könnte mit anderen Geräten verbessert werden, die eine genauere Messung ermöglichen. 
Eine weitere Möglichkeit der Bestimmung der Suszeptibilität ist die über die Brücken- und Speisespannung, welche hier aufgrund fehlender Messdaten nicht möglich war. Aus diesem Grund kann nicht gesagt werden, ob diese Methode zu besseren Ergebnissen
führt oder welche Methode die genauere ist. Es ist allerdings anzunehmen, dass aufgrund des identischen Versuchaufbaus und -durchführung ähnlich große Fehler aufgetreten wären. 