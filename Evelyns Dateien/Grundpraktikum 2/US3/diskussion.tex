\section{Diskussion}

Im ersten Versuchsteil ist zu erkennen, dass der Großteil der Messdaten aller Prismenwinkel auf einer Geraden liegen, was für einen linearen Zusammenhang zwischen dem Quotienten $\frac{\lvert \Delta \nu \rvert}{cos(\alpha)}$ und der Strömungsgeschwindigkeit 
$v$ spricht. Allerdings fällt auf, dass einige Werte der dritten Messreihe, bei einem Prismenwinkel von $\theta = 60°$, von den restlichen Messwerten stark abweichen. Diese könnte auf Ungenauigkeiten in der Messung zurückzuführen sein.
Die gemessenen Werte der Frequenzverschiebung $\Delta \nu$ entsprechen den vorherigen Vermutungen: Die Verschiebung erhöht sich für zunehmenden Prismenwinkel und ist für einen Winkel von $\theta = 30°$ stets negativ. Zudem verringert sich auch wie erwartet die
Strömungsgeschwindigkeit für zunehmenden Durchmesser.

Bei den Abbildungen \ref{fig:profil1} und \ref{fig:profil3} des zweiten Versuchsteils lässt sich feststellen, dass die Kurven parabelförmig verlaufen und so deutlich das Strömungsprofil zeigen. 
Die Abbildungen \ref{fig:profil2} und \ref{fig:profil4} zeigen in etwa den inversen Verlauf zu den vorherigen Kurven. Dies bedeutet, dass bei höherer Strömungsgeschwindigkeit die Streuintensität sinkt und damit ein besseres Signal möglich ist. 

Da die vom Rechner ausgegebenen Werte der Frequenzverschiebung und der Streuintensität stark schwanken, ist es nicht möglich sehr genaue Aussagen zu machen. Dies gilt vor allem für die Streuintensitäten, welche meist grob abgeschätzt werden müssen. Unter 
Umständen ist es möglich, dies mit einem verbesserten Versuchsaufbau zu verbessern, allerdings konnten auch in diesem Versuch qualitative Aussagen gemacht werden.