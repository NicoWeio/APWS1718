\section{Durchführung 502}

\subsection{Bestimmung der spezifischen Ladung der Elektronen}

Zur Bestimmung der spezifischen Ladung $\frac{e_0}{m_0}$ wird durch ein Helmholtzspulenpaar mit der Windungszahl $N$, dem Spulenstrom $I$, dem Spulenradius $R$ und der magnetischen Feldkonstante 
$\mu_0 = 4\pi \cdot 10^{-7}\,\symup{\frac{V s}{A m}}$ ein annähernd homogenes Magnetfeld aufgebaut. Dieses Magnetfeld besitzt in seinem Mittelpunkt eine Flussdichte von

\begin{equation}
B = \mu_0 \frac{8}{\sqrt{125}} \frac{N I}{R}.
\end{equation}

Die Kathodenstrahlröhre wird so gedreht, dass ihre Achse parallel zu der Horizontalkomponente des Erdmagnetfeldes liegt. Als Hilfsmittel wird dafür ein Deklinatorium-Inklinatorium verwendet. Danach wird eine
Beschleunigungsspannung $U_{\text{B}}$ zwischen 250 und 500\,V eingestellt und dafür die Strahlverschiebung $D$ in Abhängigkeit von $B$ gemessen, indem der Leuchtfleck auf die unterste oder oberste Linie 
des Leuchtschirms gebracht wird. Dies wird für vier weitere Beschleunigungsspannungen wiederholt.



\subsection{Bestimmung der Intensität des lokalen Erdmagnetfeldes}

Zunächst wird eine konstante Beschleunigungsspannung zwischen 150 und 200 V gewählt. Mit Deklinatorium wird die Nord-Süd-Richtung des Erdmagnetfeldes bestimmt, nach der die Achse der Kathodenstrahlröhre 
ausgerichtet wird. Es muss auf den Ort des Leuchtflecks auf dem Schirm geachtet werden, bevor die Röhre in Ost-West-Richtung gedreht wird. Die durch die Drehung im Erdmagnetfeld verursachte Verschiebung des 
Leuchtflecks kann dann mithilfe des Magnetfeldes des Helmholtzspulenpaares korrigiert werden. Dazu wird der Spulenstrom $I_{\text{hor}}$ verändert, bis sich der Leuchtfleck in der Ursprungsposition befindet. 
Der benötigte Spulenstrom wird daraufhin notiert.

Im letzten Versuchsteil wird der Inklinationswinkel $\phi$ zur Bestimmung der Totalintensität $B_{\text{total}}$ ermittelt. Hierzu muss die Winkelscheibe des Inklinatoriums so gedreht werden, dass ihre Flächennormale
parallel zum Boden liegt. Dann kann mithilfe der Kompassnadel der Winkel auf der Scheibe abgelesen werden.