\subsection{Diskussion}

Im ersten Versuchsteil wurden die Absorptionskoeffizienten des Blei- und des Eisenabsorbers unter Verwendung eines $\gamma$-Strahlers auf experimentelle Weise sowie auch theoretisch berechnet. Die Rechnungen liefern folgende Ergebnisse:

\begin{table}[htbp]
\centering
\caption{Vergleich der Absorptionskoeffizienten.}
\label{tab:litwerte}
\begin{tabular}{S S[table-auto-round, table-format=3.3] S[table-auto-round, table-format=2.2]}
\toprule
& {$\symup{Experimentell}$} & {$\symup{Theoretisch}$} \\
\midrule
$\text{Blei}$  & 105.998 $\frac{1}{\text{m}}$ & 69.19 $\frac{1}{\text{m}}$ \\
$\text{Eisen}$ & 47.86 $\frac{1}{\text{m}}$ & 56.5 $\frac{1}{\text{m}}$ \\

\bottomrule
\end{tabular}
\end{table}
Vor allem bei dem Bleiabsorber ist eine große Abweichung zwischen dem experimentellen und dem theoretischen Absorptionskoeffizientenzu erkennen. Aus diesem Grund ist davon auszugehen, dass neben dem Comptoneffekt auch der Photoeffekt 
stattfindet und im Versuch nachzuweisen ist. Die geringen Fehler der experimentell bestimmten Koeffizienten sprechen zudem für eine genaue Messung. Beim Eisenabsorber ist die Differenz der beiden Koeffizienten kleiner, weshalb der 
Photoeffekt wohl nur eine geringere Rolle spielt. Da die Paarbildung erst bei hohen Energien stattfindet, kann eine Auswirkung dieses Effekts auf die experimentellen Werte ausgeschlossen werden.

Bei der Untersuchung der Absorption von $\beta$-Strahlung wurde eine Gesamtenergie von

\begin{equation*}
E_{\text{max}} = (0{,}29 \pm 0{,}03)\,\symup{MeV}.
\end{equation*}
berechnet. Der Literaturwert des verwendeten Strahlers Technetium-99 liegt bei $E_{\text{max,lit}} = 0{,}29\,\symup{MeV}$ \cite{wiki}. Da der experimentell bestimmte Wert gerundet wurde und einen Fehler besitzt, beträgt die Abweichung von dem Literaturwert 
nicht genau 0\%, muss aber dennoch sehr klein sein, woraus folgt, dass die Messung genau war.