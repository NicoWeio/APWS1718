\section{Zielsetzung}
Das Ziel des Versuchs ist auf drei experimentelle Arten das magnetische Moment eines Permanentmagneten auszurechnen und diese anschließend zu vergleichen. 
\section{Theorie}
In dem Magnetismus wird die einfachste Form eines magnetischen Dipols beobachtet. Es gibt keine magnetischen Monopole.
Ein magnetischer Dipol besteht aus einem Permanentmagneten oder kann z.B. vom Strom durch Leiter durchflossen sein. 
Das magnetische Moment ist gegeben durch:
\begin{equation}
\vec{\mu}=I\cdot\vec{A}
\end{equation}
wobei $A$ die Querschnittsfläche der Schleife und $I$ der Strom, der durch die Leiterschleife fließt, sind.
Für ein homogenes Magnetfeld werden in der Regel Helmholtz-Spulen verwendet. Es sind zwei kreisförmige Spulen, vom Strom $I$ durchflossen und so angeordnet, dass das Magnetfeld $B$ im Zentrum der beiden entstehen kann. 
Das Magnetfeld errechnet sich dann aus:
\begin{equation}
B=\frac{\mu_{0}\cdot I\cdot R^{2}}{(R^{2}+x^{2})^{3/2}}.
\label{eq:flussdichte}
\end{equation}
\subsection{Betrachtung unter Ausnutzung der Gravitation}
Wird eine Masse mit der Gravitationskraft $\vec{F}_{g}$ an einem Drehort befestigt, so ergibt sich ein Drehmoment $\vec{D}_{\text{g}}$ mit:
\begin{equation}
\vec{D}_{\text{g}}=m\cdot(\vec{r}\times\vec{g})
\end{equation}
wobei $\vec{r}$ der Abstand von der Masse $m$ zum  Angriffspunkt der Kraft ist.
Es entsteht ein Gleichgewicht,zwischen dem Drehmoment $\vec{D}_{\text{g}}$ und dem Drehmoment $\vec{D}_{\text{B}}$ eines homogenen Magnetfeldes gegeben als:
\begin{equation}
\vec{D}_{B}= \vec{\mu}_{\text{Dipol}} \times \vec{B}
\end{equation}
Weil $g$ und $B$ parallel sind, ergibt sich das magnetische Moment ${\mu_{\text{Dipol}}}$ als:
\begin{equation}
\mu_{\text{Dipol}}=\frac{m\cdot r\cdot g}{B}.
\label{eq:udipol}
\end{equation}
\subsection{Betrachtung unter Ausnutzung der Schwingungsdauer}
In einem homogenen Magnetfeld wird die Schwingung eines Magneten wie ein harmonischer Oszillator betrachtet. Dabei wird diese Bewegung gegeben durch eine Differentialgleichung $2$. Ordnung und lautet:
\begin{equation}
-\vert\vec{\mu}_{\text{Dipol}}\times\vec{B}\vert = J_{K} \cdot \frac{d^{2}\theta}{dt^{2}}.
\end{equation} 
Dabei hängt die Lösung der Differentialgleichung von der Magnetfeldstärke $B$, dem Trägheitsmoment $J_{\text{K}}$ ab und wird über die Schwingungsdauer $T$ bestimmt als:
\begin{equation}
T^{2}=\frac{4\pi^{2}J_{K}}{B\mu_{\text{Dipol}}}.
\label{eq:Schwingungsdauer}
\end{equation}
\subsection{Betrachtung unter Ausnutzung der Präzession}
Als Präzessionbewegung bezeichnet man die Bewegung der Drehachse eines starren Körpers auf eine zweite Achse, wenn eine äußere Kraft wirkt. Diese Art von Bewegung wird durch eine Differentialgleichung beschrieben als:
\begin{equation}
\vec{\mu} \times \vec{B} = \frac{d\vec{L}_{K}}{dt}
\end{equation}  
Mit der Präzessionsfrequenz $\Omega_{\text{p}}$ als Lösung der Differentialgleichung:
\begin{equation}
\Omega_{\text{p}}=\frac{\mu B}{\vert L_{K} \vert}
\end{equation}
wobei $L_{\text{K}}$ der Drehimpuls des Körpers ist.
Für diesen Fall ergibt sich für das magnetische Moment $\mu_{\text{Dipol}}$:
\begin{equation}
\mu_{\text{Dipol}}=\frac{2\pi L_{K}}{BT_{p}}.
\label{eq:Präzession}
\end{equation}
mit $T_{\text{p}}$ als Umlaufdauer.
