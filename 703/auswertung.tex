\section{Auswertung}
\label{sec:Auswertung}

\subsection{Bestimmung der Steigung des Plateau-Bereichs}

In der Tabelle \ref{tab:plateau} befinden sich die gemessenen Werte. Der absolute Fehler für N Entladungen ergibt sich über:
\begin{equation}
\label{eqn:absoluterfehler}
\sigma_\text{N} = \sqrt{N}
\end{equation}
und für die Impulsrate:
\begin{equation}
Z = \frac{N}{T}
\end{equation} 
ergibt sich dann:
\begin{equation}
\label{eq:sigmaz}
\sigma_\text{Z} = \frac{\sqrt{N}}{T},
\end{equation}
da die Anzahl von Zerfällen poissonverteilt ist. 

Die aufgeführten Werte mit Fehlern aus der Tabelle \ref{tab:plateau} und die Ausgleichsgerade werden in die Abbildung \ref{fig:plateau} graphisch eingetragen. Zur Bestimmung der Ausgleichsgerade des Plateau-Bereichs werden die Werte ab $\SI{320}{\volt}$ bis $\SI{700}{\volt}$ verwendet. Dabei werden die Impulse in die Impulserate umgerechnet und gegen die anliegende Spannung aufgetragen(s.Abb.\ref{fig:plateau}). Die ersten zwei Werte aus der Tabelle werden nicht betrachtet, da sie sich noch nicht im Geiger-Müller-Plateau befinden.

\begin{figure}[h!]
	\centering
	\includegraphics[width=0.8\linewidth]{../../Plateau}
	\caption{Aufgenommenen Werte zur Charakteristik des Zählrohres mit Fehlern und Plateaugerade.}
	\label{fig:plateau}
\end{figure}

Eine lineare Ausgleichsgerade lässt sich berechnen wie:
\begin{equation}
\label{eq:ausgleichsgerade}
y = mx + b
\end{equation}
wobei $m$ die Steigung und $b$ der y-Achsenabschnitt sind. Über Formel \ref{eq:ausgleichsgerade} wird die Steigung und der Fehler der Ausgleichsgerade vom Python-Modul Scipy curve\_fit berechnet.
Es ergaben sich für die Ausgleichsgerade $m$ als Steigung und $b$ als y-Achsenabschnitt die folgenden Werte:
\begin{align*}
m &= \SI{0,052 \pm 0,006}{\frac{1}{\volt\second}} \\
b &= \SI{174,3 \pm 3,33}{\frac{1}{\second}}
\end{align*}
und die Plateau-Steigung wird in $\si{\percent}$ pro $\SI{100}{\volt}$ angegeben:
\begin{equation*}
m = \SI{2,2 \pm 1,7}{\percent} \text{pro 100V}.
\end{equation*}

\begin{table}[htpb]
	\centering
	\caption{Messdaten zur Charakteristik des Zählrohres.}
	\label{tab:plateau}
	\begin{tabular}{c c c c c c}
		\toprule
		$U / \si{\volt}$ & $N / \si{\frac{1}{\text{min}}}$ & $I / \si{\micro\ampere}$ & $T / \si{\second}$ & $\sigma_\text{N} $ & $\sigma_Z / \si{\frac{1}{\second}}$ \\
		\midrule
	     320 &	11428&	0,1	&60&	107	&1,8 \\
	     330&	11709&	0,2	&60	&108	&1,8\\
	     340&	11461&	0,25&	60&	107	&1,8\\
	     350&	11600&	0,3&	60&	108	&1,8\\
	     360&	11597&	0,35&	60&	108	&1,8\\
	     370&	11804&	0,4&	60	&109&1,8\\
	     380&	11843&	0,4	&60	&109	&1,8\\
	     390&	11946&	0,45&	60&	109 &1,8\\
	     400&	11731&	0,5	&60	&108    &1,8\\
	     410&	11755&	0,55&	60&	108  &1,8\\
	     420&	11839&	0,6	&60	&109		&1,8\\
	     430&	11814&	0,65&	60&	109	&1,8\\
	     440&	11999&	0,7	&60	&110		&1,8\\
	     450&	11960&	0,75&	60&	109	&1,8\\
	     460&	11923&	0,8	&60&	109   &1,8\\
	     470&	12153&	0,85	&60&110&1,8\\
	     480&	11970&	0,85	&60&109&1,8\\
	     490&	11949&	0,9&	60	&109&1,8\\
	     500&	11875&	1&	60	&109&1,8\\
	     510&	11932&	1,05&	60&	109&1,8\\
	     520&	11928&	1,1	&60	&109&1,8\\
	     530&	11954&	1,1	&60	&109&1,8\\
	     540&	12090&	1,2	&60	&110&1,8\\
	     550&	12158&	1,25&	60&	110&1,8\\
	     560&	11941&	1,3	&60	&109&1,8\\
	     570&	12040&	1,3&	60	&110&1,8\\
	     580&	12009&	1,3	&60	&110&1,8\\
	     590&	11995&	1,3	&60	&1110&1,8\\
	     600&	11952&	1,4	&60	&109&1,8\\
	     610&	12129&	1,45&	60&	110,&1,8\\
	     620&	12003&	1,5	&60	&110&1,8\\
	     630&	12516&	1,5	&60	&112&1,9\\
	     640&	12263&	1,55&	60&	111&1,8\\
	     650&	12324&	1,6	&60&	111&1,9\\
	     660&	12584&	2	&60&	112&1,9\\
	     670&	12692&	1,8	&60&	113&1,9\\
	     680&	12911&	1,8	&60&	114&1,9\\
	     690&	13075&	2	&60&	114&1,9\\
	     700&	13765&	2,1	&60&	117&2,0\\	     
	\end{tabular}
\end{table}

\subsection{Bestimmung der Totzeit}
\subsubsection{Mit dem Oszilloskop}

Für die erste Methode zur Bestimmung der Totzeit wird das Ablesen auf dem Oszilloskop benötigt. Diese Messung wird gemäß Abbildung \ref{fig:zeiten} durchgeführt und die aufgeführten Messwerte wie die Spannung $U$, die Totzeit $T$ und Erholungszeit $T_\text{E}$ befinden sich in der Tabelle \ref{tab:erstemethode}.

\begin{table}[htpb]
	\centering
	\caption{Messwerte zur ersten Methode für die Bestimmung der Totzeit.}
	\label{tab:erstemethode}
	\begin{tabular}{c c c}
		\toprule
		$U / \si{\volt}$ & $T / \si{\micro\second}$ & $T_\text{E} / \si{\micro\second}$ \\
		\midrule
		460 & 250 & 490 \\
		500 & 270 & 430 \\
		550 & 275 & 480 \\
		\bottomrule
	\end{tabular}
\end{table}

Der Mittelwert $\bar{x}$ aus $n$ Stichproben $x_{i}$ ergibt sich aus:
\begin{equation}
\bar{x}=\frac{1}{n}\sum \limits_{i=1}^n x_{i}
\label{eq:mittelwert}
\end{equation}
Die Standardabweichung errechnet sich nach:
\begin{equation}
\label{eq:standardab}
s_{i}= \sqrt{\frac{1}{n-1}\sum \limits_{i=1}^n(x_{i}-\bar{x})^2}
\end{equation}
mit zufälligen Fehlern behafteten Werten $x_{i}$ mit $i$ = 1,...,n.\\
Der aus der Standardabweichung aus der Gleichung \ref{eq:standardab} resultierende Fehler des Mittelwertes ergibt sich nach:
\begin{equation}
\Delta\bar{x}=\frac{s_{i}}{\sqrt{n}} = \sqrt{\frac{\sum\limits_{i=1}^n(x_{i}-\bar{x})^2}{n(n-1)}}
\label{eq:fehlermittelwert}
\end{equation}

Mit der Gleichung \ref{eq:mittelwert} und \ref{eq:fehlermittelwert} soll nun der Mittelwert sowie der Fehler des Mittelwerts der Totzeit und Erholungszeit bestimmt werden und es ergeben sich die folgenden Werte:
\begin{align*}
T_\text{tot1} &= \SI{265 \pm 7}{\micro\second} \\
T_\text{E} &= \SI{466,6 \pm 18,5}{\micro\second}.
\end{align*}

\subsubsection{Mit der Zwei-Quellen-Methode}
Zur zweiten Methode für die Bestimmung der Totzeit ergaben sich bei einer Spannung $U$ von $\SI{500}{\volt}$ die in Tabelle \ref{tab:zweitemethode} aufgeführten Werte. Die Totzeit wird nun nach der Gleichung \ref{eq:totzeit} berechnet und der Fehler wird nach Gaußscher Fehlerfortpflanzung \ref{eqn:gauß} gegeben:

\begin{equation}
\label{eqn:gauß}
\sigma_T = \sqrt{ \left(\frac{N_{1+2}-N_{2}}{2N_{1}^{2}N_{2}}\right)^2 \cdot \sigma_{N_{1}}^2+ \left(\frac{N_{1+2}-N_{1}}{2N_{1}N_{2}^{2}}\right)^2 \cdot \sigma_{N_{2}}^2+\left(\frac{1}{2N_1N_2}\right)^2 \cdot \sigma_{N_{1+2}}^2} 
\end{equation}

Damit ergibt sich für die Totzeit der folgende Wert:
\begin{equation*}
T_\text{tot2} = \SI{932,2 \pm 92,1}{\micro\second}.
\end{equation*}

\begin{table}[htpb]
	\centering
	\caption{Messwerte zur zweiten Methode für die Bestimmung der Totzeit.}
	\label{tab:zweitemethode}
	\begin{tabular}{c c c c c c}
		\toprule
		Quellen & $U / \si{\volt}$ & $N / \si{\frac{1}{\second}}$ & $\sigma_\text{Z} / \si{\frac{1}{\second}}$ & $\sigma_\text{N}$ & $T / \si{\second}$ \\
		\midrule
		1& 500 & 12120 & 1,83 & 110,1 & 60 \\
		1+2 & 500 & 13038 & 1,90 & 114,18 & 60 \\
		2 & 500 & 1186 & 0,57 & 34,43 & 60 \\
		\bottomrule
	\end{tabular}
\end{table}

\subsection{Bestimmung der Ladung pro einfallendem Teilchen}
Die Ladungsmenge, welche durch ein Teilchen freigesetzt wird, hängt vom gemessenen Strom ab. Dieser ist in Tabelle \ref{tab:plateau} bereits eingeführt worden. 

Die Ladung $Q$ pro Teilchen ergibt sich durch Gleichung \ref{eq:ladung}, die nach der Ladung $Q$ umgestellt wird:
\begin{equation}
Q = \frac{I \cdot T}{N}
\end{equation}
wobei $Z$ die Impulsrate, $T$ die Messzeit und $I$ die Stromstärke sind. Die Fehler der Ladung berechnet sich wie folgt:
\begin{equation}
\label{eq:fehlerladung}
\sigma_{Q} = Q \cdot \frac{\sigma_Z}{N}
\end{equation}
Die Abweichung für die Ladung errechnet sich nach:
\begin{equation}
\sigma_{Q1} = \SI{6,24 e18}{\electronvolt}\cdot \sigma_{Q}
\end{equation}
Es lässt sich ein linearer Zusammenhang zwischen der angelegten Spannung $U$ und Ladung $Q$ pro im Zählrohr einfallendem Teilchen feststellen. Dieser Zusammenhang inklusive der Ausgleichsgerade ist in Abbildung \ref{fig:ladung} dargestellt. 

Über Formel \ref{eq:ausgleichsgerade} wird die Steigung und der Fehler der Ausgleichsgerade vom Python-Modul Scipy curve\_fit berechnet.
Es ergaben sich für die Ausgleichsgerade $m$ als Steigung und $b$ als y-Achsenabschnitt die folgenden Werte:
\begin{align*}
m &= \SI{0,138 \pm 0,002 e9}{\frac{\elementarycharge}{\volt}} \\
b &= -\SI{39,09 \pm 1,35 e9}{\elementarycharge}.
\end{align*}

\begin{figure}[h!]
	\centering
	\includegraphics[width=0.8\linewidth]{../../Ladung}
	\caption{Ausgleichsgerade zur Bestimmung der Ladung mit den zugehörigen Fehlern.}
	\label{fig:ladung}
\end{figure}

\begin{table}[htpb]
	\centering
	\caption{Messdaten zur Charakteristik des Zählrohres.}
	\label{tab:ladung}
	\begin{tabular}{c c c c c c c c c c}
		\toprule
		$U / \si{\volt}$ & $N / \si{\frac{1}{\second}}$ & $I / \si{\micro\ampere}$ & $T / \si{\second}$ & $Z / \si{\frac{1}{\second}}$ & $\sigma_\text{Z} / \si{\frac{1}{\second}}$ & $Q / \SI{e-9}{\coulomb}$ & $\sigma_{Q} / \SI{e-9}{\coulomb}$ & $Q / \SI{e9}{\elementarycharge}$ & $\sigma_{Q1} / \SI{e9}{\elementarycharge}$ \\
		\midrule
		320	&11428&	0,1 &60&190,4&1,8&5,25&0,000081&3,2&0,00051 \\
		330	&11709&	0,2	&60&195,1&1,8&1,02&0,00015&6,3&0,00098 \\
		340	&11461&	0,25&60&191,1&1,8&1,30&0,00020&8,1&0,0012\\
		350	&11600&	0,3	&60&193,3&1,8&1,55&0,00024&9,6&0,0014\\
		360	&11597&	0,35&60&193,3&1,8&1,81&0,00028&11,3&0,0017\\
		370	&11804&	0,4	&60&196,7&1,8&2,03&0,00031&12,6&0,0019\\
		380	&11843&	0,4	&60&197,4&1,8&2,02&0,00031&12,6&0,0019\\
		390	&11946&	0,45&60&199,1&1,8&2,26&0,00034&14,1&0,0021\\
		400	&11731&	0,5	&60&195,5&1,8&2,55&0,00039&15,9&0,0024\\
		410	&11755&	0,55&60&195,9&1,8&2,80&0,00043&17,5&0,0026\\
		420	&11839&	0,6	&60&197,3&1,8&3,04&0,00046&18,9&0,0029\\
		430	&11814&	0,65&60&196,9&1,8&3,30&0,00050&20,6&0,0031\\
		440	&11999&	0,7	&60&199,9&1,8&3,50&0,00053&21,8&0,0033\\
		450	&11960&	0,75&60&199,3&1,8&3,76&0,00057&23,4&0,0035\\
		460	&11923&	0,8	&60&198,7&1,8&4,02&0,00061&25,1&0,0038\\
		470	&12153&	0,85&60&202,5&1,8&4,19&0,00063&26,1&0,0039\\
		480	&11970&	0,85&60&199,5&1,8&4,26&0,00064&26,5&0,0041\\
		490	&11949&	0,9	&60&199,1&1,8&4,51&0,00068&28,2&0,0043\\
		500	&11875&	1	&60&197,9&1,8&5,05&0,00077&31,5&0,0048\\
		510	&11932&	1,05&60&198,8&1,8&5,27&0,00084&32,9&0,0050\\
		520	&11928&	1,1	&60&198,8&1,8&5,53&0,00084&34,5&0,0052\\
		530	&11954&	1,1	&60&199,2&1,8&5,52&0,00090&34,4&0,0052\\
		540	&12090&	1,2	&60&201,5&1,8&5,95&0,00093&37,1&0,0056\\
		550	&12158&	1,25&60&202,6&1,8&6,16&0,00099&38,5&0,0058\\
		560	&11941&	1,3	&60&199,1&1,8&6,53&0,00098&40,7&0,0062\\
		570	&12040&	1,3	&60&200,6&1,8&6,47&0,00098&40,4&0,0061\\
		580	&12009&	1,3	&60&200,1&1,8&6,49&0,00097&40,5&0,0061\\
		590	&11995&	1,3	&60&199,9&1,8&6,50&0,00098&40,5&0,0061\\
		600	&11952&	1,4	&60&199,2&1,8&7,02&0,00107&43,8&0,0066\\
		610	&12129&	1,45&60&202,1&1,8&7,17&0,00108&44,7&0,0067\\
		620	&12003&	1,5	&60&200,0&1,8&7,49&0,00114&46,8&0,0071\\
		630	&12516&	1,5	&60&208,6&1,9&7,19&0,00107&44,8&0,0066\\
		640	&12263&	1,55&60&204,3&1,8&7,58&0,00114&47,3&0,0071\\
		650	&12324&	1,6	&60&205,4&1,9&7,78&0,00116&48,6&0,0072\\
		660	&12584&	2	&60&209,7&1,9&9,53&0,00141&59,5&0,0088\\
		670	&12692&	1,8	&60&211,5&1,9&8,50&0,00125&53,1&0,0078\\
		680	&12911&	1,8	&60&215,1&1,9&8,36&0,00122&52,1&0,0076\\
		690	&13075&	2	&60&217,9&1,9&9,17&0,00133&57,2&0,0083\\
		700	&13765&	2,1	&60&229,4&2,0&9,15&0,00130&57,1&0,0081\\
		 
	\end{tabular}
\end{table}
