\section{Diskussion}
\label{sec:Diskussion}
Um ein perfektes Geiger-Müller-Zählrohr zu konstruieren wäre eine Plateausteigung von $m = \SI{0}{\frac{\percent}{\volt}}$ erstrebenswert. Das ist natürlich nur in der Theorie möglich , aber das Ergebnis dieser Messung zeigt, dass die errechnete Steigung von $m = \SI{2,2 \pm 1,7}{\percent} \text{pro 100V}$ in einem Bereich liegt, der für durchschnittliche Zählrohre passabel ist. Es ist auch noch hervorzuheben, dass während der Messung die Analoganzeige des $\si{\micro\ampere}$-Meters ihren Verhältnissen entsprechend stark am Schwanken war. Ein genauer Fehler lies sich hierbei nicht bestimmen. Ein Amperemeter mit einer genaueren Auflösung hätte sich hier als hilfreich erwiesen. 

Bei der Aufnahme der Charakteristik des Zählrohrs fällt auf, dass gegen Ende die Anzahl der gemessenen Impulse, wie erwartet, wieder stärker ansteigt und so ein klares Ende das Plateaus definierbar ist. Das Ende des Plateaus wird bei 700V tatsächlich erreicht, weil die ersten zwei Messwerte vernachlässigt wurden, da sie sich noch nicht im Geiger-Müller-Plateau befinden.

Die pro Teilchen freigesetzte Ladung steht in einem linearen Zusammenhang zur angelegten Spannung. Dies lässt sich durch den proportionalen Zusammenhang zwischen der angelegten Spannung $U$ und der Stärke des dadurch entstehenden elektrischen Feldes im Zählrohr erklären. 
Für die zu bestimmende Ladung ergibt sich ein Wert von $m = \SI{0,138 \pm 0,002 e9}{\frac{\elementarycharge}{\volt}}$. Für diesen Wert lässt sich ein Vergleich in der Abbildung \ref{fig:bereiche} auf der y-Achse für die Anzahl der Elektronen-Ionen-Paare im $10^{10}$-Bereich feststellen. Es ergibt sich ein Unterschied um eine Größenordnung. Da der Strom schwierig abzulesen war, ergaben sich auch Sprünge in den Messdaten, wie in Tabelle \ref{tab:ladung} und Abbildung \ref{fig:ladung} zu erkennen ist. 

Die am Oszilloskop abgelesenen Werte sind sehr ungenau, da sie nur geschätzt werden konnten zum Beispiel beim Ablesen der Erholungszeit. 
Jedoch erwies sich eine Totzeit von $T_\text{tot1} = \SI{265}{\micro\second}$.

Ganz im Gegensatz zu der genauen Bestimmung der Totzeit mit Hilfe des Oszilloskops steht die Bestimmung anhand der Zweiquellenmethode. Diese belief sich bei diesem Experiment auf $T_\text{tot2} = \SI{932,2 \pm 92,1}{\micro\second}$. Damit zeigt sie ein großer Unterschied im Gegensatz zur ersten Methode. Dieser Wert scheint eher untypisch für ein Geiger-Müller-Zählrohr zu sein. Hier muss ein systematischer Fehler vorliegen, auf den während des Experimentes nicht geachtet wurde. Dabei liegt es auch an der Formel und den hohen Fehlern.


