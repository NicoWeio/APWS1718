\section{Diskussion}
\label{sec:Diskussion}

In der Tabelle \ref{tab:zusammenfassung} sind die relativen Fehler bei den Ergebnissen für die Berechnung der spezifischen Ladung zu sehen. Die relativen Fehler sind sehr klein und weisen darauf hin, dass die Ergebnisse ziemlich exakt sind. Der exakteste Wert liegt bei einer Beschleunigungsspannung von $U_\text{B} = \SI{400}{\volt}$ und beträgt somit der kleinste relative Fehler. Diese können nicht mit genauen Messungen erklärt werden, sondern durch eine geringe Empfindlichkeit.
Der Literaturwert beträgt $\SI{1,759E11}{\frac{\coulomb}{\kilogram}}$ \cite{spezifischeElektronenladung}. Die Abweichung vom Literaturwert zu den berechneten Mittelwert der spezifischen Ladung aus der Gleichung \ref{eq:mittelwertspez} ist mit $\SI{12,79}{\percent}$ behaftet. Auch hier ist die Abweichung relativ klein aber erreicht nicht den Literaturwert innerhalb der errechneten Ungenauigkeit. Messungenauigkeiten ergaben sich durch einen relativen dicken Leuchtpunkt. 

Bei der Bestimmung des lokalen Erdmagnetfeldes fällt das Ergebnis in die richtige Größenordnung von einigen zehn $\si{\micro\tesla}$. Beim Ablesen des Winkels mithilfe des Inklinatorium-Deklinatoriums und durch den Einfluss von störenden Magnetfelder, die vermutlicherweise durch anderen Messgeräten in den vorherigen Versuchsteilen entstanden sind, in der Umgebung der Apparatur haben sich einige Messfehler ergeben, da zum Beispiel das Messgerät zur Bestimmung des Erdmagnetfeldes ziemlich ungenau ist.
Für den Standort Dortmund ergibt sich für das Erdmagnetfeld $B_\text{theo} = \SI{49,4}{\micro\tesla}$ und für den Inklinationswinkel $\phi = 66$ $\circ$ \cite{Deklinationscalc}. Durch einen Vergleich mit dem experimentellen und theoretischen Wert liegt die Abweichung bei $\SI{2,02}{\percent}$ für das Erdmagnetfeld und $\SI{6,21}{\percent}$ für den Inklinationswinkel. Somit ist das Ergebnis fast genau, liegt sehr nah an der Realität und entspricht den Erwartungen. 

Zu den jeweiligen Abbildungen der Ausgleichsgeraden ist zu sagen, dass sie zu den Messwerten übereinstimmen und entsprechen den Erwartungen. 

\begin{table}[htbp]
	\centering
	\caption{Die errechneten Steigungen sowie der relative Fehler in Prozent.}
	\label{tab:zusammenfassung}
	\begin{tabular}{c c c}
		\toprule
		$U_\text{B} / \si{\volt}$ & $\frac{e_0}{m_0} / \cdot 10^{11}  \si{\frac{\coulomb}{\kilogram}}$ & Relative Fehler $ / \si{\percent}$\\
		\midrule
	    250 & 1,485 $\pm$ 0,007 & 0,471 \\
	    300 & 1,779 $\pm$ 0,008 & 0,449 \\
	    350 & 1,064 $\pm$ 0,012 & 1,127 \\
	    400 & 1,749 $\pm$ 0,001 & 0,057 \\
	    440 & 1,594 $\pm$ 0,005 & 0,313 \\
		\bottomrule
	\end{tabular}
\end{table}
\FloatBarrier


