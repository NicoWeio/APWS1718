\section{Diskussion}
\label{sec:Diskussion}
Zum ersten Teil des Versuches ergaben sich Probleme bei hohen Beschleunigungspannungen, da nicht genug Ablenkspannung angelegt werden konnte, um den Strahl über den kompletten Schirm zu bewegen. Bei den Ergebnissen für die Steigung in der Tabelle \ref{tab:zusammenfassung} ist zu sehen, dass die Empfindlichkeit mit zunehmender Beschleunigungsspannung deutlich abnimmt, also die Ablenkung wird kleiner. Der genauste Wert ist bei einer Beschleunigungsspannung $U_\text{B} = \SI{350}{\volt}$, hat die kleinste relative Fehler und hat auch die geringste Empfindlichkeit. Die Abbildungen zu den Ausgleichsgeraden entsprechen auch den Erwartungen und sind ziemlich exakt. 

\begin{table}[htbp]
	\centering
	\caption{Die errechneten Steigungen sowie der relative Fehler in Prozent.}
	\label{tab:zusammenfassung}
	\begin{tabular}{c c c}
		\toprule
		$U_\text{B} / \si{\volt}$ & $\frac{e_0}{m_0} /  \si{\frac{\cm}{\volt}}$ & Relative Fehler $ / \si{\percent}$\\
		\midrule
		220 & 0,1524 $\pm$ 0,0011 & 0,721 \\
		300 & 0,1193 $\pm$ 0,0007 & 0,586 \\
		350 & 0,0998 $\pm$ 0,0005 & 0,501 \\
		400 & 0,0842 $\pm$ 0,0008 & 0,950 \\
		450 & 0,0786 $\pm$ 0,0007 & 0,890 \\
		\bottomrule
	\end{tabular}
\end{table}
\FloatBarrier

Bei der Bestimmung der Apparaturkonstante sollte nun die Größe $\frac{pL}{2D}$ mit der errechneten Steigung verglichen werden. Es ergibt sich nun eine Ungenauigkeit von $\SI{46,13}{\percent}$ zum theoretischen Wert und eine relative Fehler von $\SI{4,36}{\percent}$. Somit stellt sich heraus, dass es sich eine große Abweichung ergeben hat, die möglicherweise dadurch erklärt werden kann, dass die Ablenkplatten aus der Konstruktionszeichnung in der Anleitung nicht mit den verwendeten Kathodenstrahlröhre aus dem Versuch übereinstimmen. Die Ablenkplatten der Kathodenstrahlröhre sind nur zum Teil parallel und verlaufen linear auseinander. 

Zur Wechselstromfrequenz des Sinusgenerators zeigten sich große Schwankungen bei dem Frequenzzähler, somit ergaben sich Fehler beim Ablesen der Frequenz. Der Mittelwert der Frequenz $\nu = \SI{78,93 \pm 0,00}{\Hz}$ liegt sehr nah an der Realität beziehungsweise an dem vom Hersteller eingegrenzte Frequenz. Da für den Wert der errechneten Scheitelspannung keine genaue Fehler angegeben werden kann, muss zumindest eine Fehler bei der Empfindlichkeitsmessung vorliegen. Die Abbildung \ref{fig:3070380616627317904748572025993656149737472n} entspricht den Erwartungen und Voraussetzungen für die Entstehung einer stehenden Welle. 

