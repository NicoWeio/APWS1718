\section{Diskussion}
\label{sec:Diskussion}
Das Experiment weist bei der Anwendung einige Fehlerquellen auf. Zum einen gibt es in dem Raum, in dem der Versuch durchgeführt wird, neben der Quecksilber-Lampe auch andere Lichtquellen, die zu Streulicht im Prisma führen können. Dadurch können die Messwerte verfälscht werden.
Es ist nicht auszuschließen, dass kleine Verunreinigungen auf der Oberfläche des Prismas den Strahlengang verändern.
Bei den unterschiedlichen Spektrallinien war das Fadenkreuz nicht gut zu sehen durch die Helligkeit der Linien und da diese Spektralinien eine endliche Breite besitzen, konnte das Fadenkreuz nicht exakt auf die Mitte gerichtet werden. 
Es ergaben sich auch Ablesefehler beim Übereinanderlegen des reflektierten und gebrochenen Strahles, da der reflektierte Strahl nur sehr schwach erscheint.

Die Bestimmung des brechenden Winkels zeigt, dass dieser eine relativ kleine Fehler besitzt und dieser ein Winkel von $\varphi = 60^\circ$ aufzeigt. Dies kann auch auf die Qualität der Apparatur führen. Außerdem ist es noch zu erkennen, dass in der Tabelle \ref{tab:brechungswinkel2} der Winkel $\eta$ mit sinkender Wellenlänge immer mehr zunimmt. Die Summe der Abweichungsquadrate zeigt, dass die Dispersionskurve \ref{fig:dispersionskurve} im sichtbaren Wellenlängenbereich gut mit der Ausgleichskurve \ref{eq:dispersionsgleichung} angenähert werden kann. 

Die berechnete nächstgelegene Absorptionsstelle $\lambda_1 $ ist kürzer als die Wellenlängen im sichtbaren Bereich und liegt im UV-Wellenlängenbereich. 

Durch die berechnete Abbesche Zahl von $21,62$ kann das Material des Prismas den Flintgläsern mit einer Abbeschen Zahl $\nu<50 $ zugeordnet werden, das bedeutet, dass die verwendete Prisma eine hohe Dispersion besitzt im Gegensatz zu den Krongläsern($\nu>50$). \cite{AbbescheZahl}

Bei dem Auflösungsvermögen ist bei den Ergebnisse für $A_C$ und $A_F$ zu erkennen, dass dieses für kleine Wellenlängen größer ist im Gegensatz zu den größeren Wellenlängen. Kurzwelliges Licht wird stärker gebrochen als langwelliges.