\section{Diskussion}
\label{sec:Diskussion}
Zu dem Versuchsteil zur Bestimmung der Lage und Lochbreite für den B-Scan ist zu erkennen, dass die Reichweite der Ultraschallwelle mit zunehmender Frequenz abnimmt. Jedoch nimmt auch die Schärfe der dargestellten Störstellen zu. Es kann gesagt werden, dass hohe Frequenzen zwar wenig tief eindringen, doch im nahen Bereich sehr gute Auflösung erreichen, während mit weniger hohen Frequenz mehr ein Gesamtüberblick erreicht wird, doch keine genauen Aussagen getroffen werden können. 

Beim B-Scan gibt es wenige bis keine weißen/scharfen Stellen. Es liegen keine tatsächlichen Werte der Durchmesser der Löcher vor, weshalb die ermittelten Ergebnisse schlecht bewertet werden können wie bei dem A-Scan Verfahren. Es fällt auf dass die Werte bei den Bohrungen $6$ bis $10$ ungenau sein könnten, da sie nicht in die Reihe der anderen Messwerte passen, die mit steigender Lochnummer im Durchmesser sinken. Vermutlich wurde hier ungenau gemessen, wodurch die geringen Unterschiede beim Lochdurchmesser nicht erkennbar sind.

Zur besseren Unterscheidbarkeit einzelner Bohrungen in den Bildern hätte jeweils eine Farbeinteilung eingefügt werden müssen, um die zu untersuchende Löcher besser unterscheiden zu können. 

Die Vermessung des Herzzeitvolumens ergab einen Wert von $\text{HZV} = \SI{1,15}{\frac{\liter}{\minute}}$. Dieser lässt sich mit einem Literaturwert\cite{Herzzeitvolumen} für das Herzzeitvolumen vergleichen, welcher sich zwischen dem Bereich von $\SI{4,5}{\frac{\liter}{\minute}}$ und $\SI{5}{\frac{\liter}{\minute}}$ befindet. Dies ist leider nicht sinnvoll, da es sich eine relative Abweichung von $\SI{74,4}{\percent}$ ergibt. Dies ist eine relativ große Abweichung zum Literaturwert.

Bei der Untersuchung des Auflösungsvermögens kann das gleiche wie bei der B-Scan gesagt werden, dass hochfrequente Sonden für den Nahbereich benutzt werden sollten, weil sie sich gut auflösen können, während niederfrequente Sonden für den Tiefscan geeignet sind, wobei sie dann auch keine hohe Auflösen besitzen. 

Für jedes Problem kann nicht dieselbe Sonde nützlich sein, sondern es sollten möglichst verschiedene Sonden genutzt werden um ein gutes Ergebnis zu erhalten.