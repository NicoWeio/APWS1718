\section{Diskussion}
\label{sec:Diskussion}

Bei der Synthese wurden leider die experimentellen Werte nicht aufgeschrieben und somit war es nicht möglich die Abweichung zu bestimmen. Jedoch wurden während der Durchführung die theoretischen Werte ausgerechnet und somit die Abweichung von der Theorie bestimmt. Es tritt eine Abweichung von maximal $\SI{8,6}{\percent}$ auf. 

Bei der Sägezahnspannung in der Abbildung \ref{fig:map018} ist die Funktion sehr gut ersichtlich. An den Unstetigkeitsstellen lässt sich gut das Gibbsche Phänomen erkennen, denn es treten starke Überschwingungen auf. An den $5$ Oberwellen lässt sich auch noch erkennen, dass diese in der Stärke des Einflusses auf die Funktion mit steigendem $n$ abnehmen. 

Bei der Dreieckspannung in der Abbildung \ref{fig:map008} ist die Schwingung sehr gut dargestellt. Hier lässt sich auch gut erkennen dass die Amplituden mit $\frac{1}{n^{2}}$ abfallen und die höheren Oberwellen kaum ins Gewicht fallen. Außerdem wurden die höheren Amplituden vernachlässigt, weil je niedriger $n$ wurde, desto schwerer war es den Wert bei den AC-Voltmeter einzustellen.

Bei der Rechteckspannung in der Abbildung \ref{fig:map005jpg} ist die Funktion mit den $5$ Oberwellen gut ersichtlich. An den Unstetigkeitsstellen sind ebenfalls starke Überschwingungen zu sehen.
Anschließend lässt sich bei der Synthese erschließen, dass es gute Ergebnisse herausgestellt haben. Das Gibbsche Phänomen ist am besten bei der Dreieckspannung und Rechteckspannung erkennbar. 

Bei der Analyse haben sich bei den Messwerten große Spannungen ergeben. Hier ergeben sich zwischen den gemessenen und theoretischen Werte kleine Abweichungen bei der Sägezahn- sowie Dreieckspannung. Bei der Rechteckspannung ist die größte Abweichung erkennbar. Bei der Fourier-Analyse entstehen Abweichungen zwischen den ermittelten Koeffizienten über die gemessenen Amplituden der Schwingungen und den theoretisch gerechneten Werten. Es ergeben sich dadurch auch systematische Fehler, denn es wird über ein unendliches Integral statt ein endliches Integral integriert. 