\newpage
\section{Auswertung}
\label{sec:Auswertung}
\subsection{Schubmodul}
\label{sec:Schubmodul}
Aus den gemessenen Periodendauern $T$ und den Formeln \eqref{eqn:Schubmodul},
\eqref{eqn:Mittelwert} und \eqref{eqn:Mittelwertfehler}
lässt sich das mittlere Schubmodul
\begin{equation*}
  \overline{G} = \SI{69.7(28)}{\giga\pascal}
\end{equation*}
bestimmen. Dabei wurde der gemittelte Drahtdurchmesser nach Formel
 \eqref{eqn:Mittelwert} und \eqref{eqn:Mittelwertfehler}
\begin{equation*}
  \overline{R} = \SI{0.102(1)}{\milli\meter}
\end{equation*}
verwendet. Die Masse der Kugel $m_\text{K} = \SI{588.3(2)}{\gram}$ und ihr
Radius $R_\text{K} = \SI{25.52(1)}{\milli\meter}$ wurden der
Versuchsapperatur entnommen. Das Trägheitsmomet der Kugelhalterung $\theta_\text{KH}$
ist auf $\SI{22.5}{\gram\centi\meter\squared}$ gegeben gewesen. Damit ergibt
sich das Gesamtträgheitsmoment $\theta$ der Kugel und ihrer Halterung zu
\begin{equation*}
  \theta = \SI{0.15545(14)}{\gram\meter\squared}.
\end{equation*}
Aus dem Schubmodul $G = \SI{69.7(28)}{\giga\pascal}$ und dem gegeben
Elastizitätsmodul \\
$E =~\SI{210.0(5)}{\giga\pascal}$,
sowie der Formel \eqref{eqn:Gauß} lässt sich der Kompressionsmodul
\begin{equation*}
  Q = \SI{-10(50)}{\tera\pascal}
\end{equation*}
nach Formel \eqref{eqn:Kompressionsmodul} und einsetzen der Formel \eqref{eqn:Querkonstante} bestimmen.
Desweiteren ergibt sich die Poissonsche Querkonstraktionszahl $\mu$ nach Formel
\eqref{eqn:Querkonstante} und \eqref{eqn:Gauß} zu
\begin{equation}
  \mu = \num{0.51(6)}.
\end{equation}
\begin{table}
  \centering
  \caption{Messwerte mit senkrecht stehendem Magneten zur Bestimmung des Schubmoduls. \cite{uncertainties}}
  \label{tab:Schubmodul}
  \begin{tabular}{S[table-format=2.3] S[table-format=3.2] @{${}\pm{}$} S[table-format=1.2]}
    \toprule
    {$T/\si{\second}$} & \multicolumn{2}{c}{$G/\si{\giga\pascal}$} \\
    \midrule
    18.474 & 69.69 & 2.85 \\
    18.474 & 69.69 & 2.85 \\
    18.477 & 69.67 & 2.84 \\
    18.476 & 69.68 & 2.84 \\
    18.473 & 69.70 & 2.85 \\
    18.474 & 69.69 & 2.85 \\
    18.472 & 69.71 & 2.85 \\
    18.473 & 69.70 & 2.85 \\
    18.472 & 69.71 & 2.85 \\
    18.473 & 69.70 & 2.85 \\
    18.474 & 69.69 & 2.85 \\
    18.472 & 69.71 & 2.85 \\
    \bottomrule
  \end{tabular}
\end{table}

\begin{table}
  \centering
  \caption{Messwerte zur Bestimmung des Drahtdurchmessers.}
  \label{tab:Drahtdurchmesser}
  \begin{tabular}{S[table-format=2] S[table-format=1.3]}
    \toprule
    {Messung} & {$d/\si{\milli\meter}$} \\
    \midrule
    1	& 0.205 \\
    2	& 0.205 \\
    3	& 0.205 \\
    4	& 0.206 \\
    5	& 0.203 \\
    6	& 0.206 \\
    7	& 0.205 \\
    8	& 0.203 \\
    9	& 0.200 \\
    10	& 0.201 \\
    11	& 0.207 \\
    12	& 0.202 \\
    \bottomrule
  \end{tabular}
\end{table}



\subsection{Magnetischer Moment}
\label{sec:MagnetischerMoment}
In der Abbildung
\begin{figure}
  \centering
  \includegraphics[width=\textwidth]{build/TB-Diagramm.pdf}
  \caption{Das B-Feld des Helmholzspulenpaares gegen $\frac{1}{\overline{T_\text{i}}^2}$ aufgetragen. \cite{matplotlib}}
  \label{fig:Plot}
\end{figure}
\ref{fig:Plot} ist das Magnetfeld des Helmholtzspulen-Paares im Mittelpunkt,
welches nach
\begin{equation}
  B = 4 \cdot \symup{\pi} \cdot 10^{-7} \frac{8 I N}{\sqrt{125} R_\text{SP}}
  \label{eqn:Helmholtz}
\end{equation} berechnet wird,
gegen den Kehrwert der gemittelten und quadrierten Periodendauer
$\frac{1}{\overline{T_\text{i}}^2}$ aufgetragen, wodurch sich eine Gerade
ergibt. In der Formel \eqref{eqn:Helmholtz} ist $N = 80$ die Anzahl der Windungen
der Spulen, $R_\text{SP} = \SI{72}{\milli\meter}$ ist der Radius der Spulen und $I$ der Strom, welcher
durch die Spulen fließt.
Die Regression wurde mit curvefit von scipy \cite{scipy} und der Funktion
\begin{equation*}
  B = a \cdot \frac{1}{T^2} + b
\end{equation*}
durchgeführt. Die Parameter ergeben sich zu
\begin{align*}
  a & = \SI{0.0946(29)}{\kilo\gram\per\ampere} & b &= \SI{-0.0002(0)}{\tesla}.
\end{align*}
Durch Koeffizientenvergleich mit der umgestellten Formel \eqref{eqn:magPeriode}
und der Formel \eqref{eqn:Gauß} ergibt sich
die Bestimmung des magnetischen Moments $m$ des eingeschraubten Magneten zu
\begin{equation*}
  m = \frac{4 \symup{\pi}^2 \theta}{a} = \SI{0.65(2)}{\ampere\meter\squared}.
\end{equation*}

\begin{table}
  \centering
  \caption{Messwerte zur Bestimmung des magnetischen Moments.}
  \label{tab:MM1}
  \sisetup{table-format=2.3}
  \begin{tabular}{S S S S S S S S S S}
    \toprule
    {$\SI{0.1}{\ampere}$} & {$\SI{0.2}{\ampere}$} & {$\SI{0.3}{\ampere}$} & {$\SI{0.4}{\ampere}$} & {$\SI{0.5}{\ampere}$} & {$\SI{0.6}{\ampere}$} & {$\SI{0.7}{\ampere}$} & {$\SI{0.8}{\ampere}$} & {$\SI{0.9}{\ampere}$} & {$\SI{1.0}{\ampere}$} \\
    {$T_{0,1}/\si{\second}$} & {$T_{0,2}/\si{\second}$} & {$T_{0,3}/\si{\second}$} & {$T_{0,4}/\si{\second}$} & {$T_{0,5}/\si{\second}$} & {$T_{0,6}/\si{\second}$} & {$T_{0,7}/\si{\second}$} & {$T_{0,8}/\si{\second}$} & {$T_{0,9}/\si{\second}$} & {$T_{1,0}/\si{\second}$} \\
    \midrule
    17.927 & 15.467 & 13.995 & 12.984 & 11.767 & 11.037 & 10.341 & 10.045 & 9.378 & 9.003 \\
    17.919 & 15.460 & 19.983 & 12.964 & 11.776 & 11.015 & 10.363 & 10.026 & 9.362 & 8.999 \\
    17.921 & 15.456 & 13.982 & 12.949 & 11.755 & 11.034 & 10.369 & 10.032 & 9.311 & 8.992 \\
    17.915 & 15.451 & 13.977 & 12.938 & 11.744 & 11.009 & 10.368 & 10.012 & 9.350 & 8.997 \\
    17.921 & 15.508 & 13.969 & 12.928 & 11.745 & 11.008 & 10.349 & 10.025 & 9.363 & 8.999 \\
    \bottomrule
  \end{tabular}
\end{table}

\begin{table}
  \centering
  \caption{Mittelwerte der Schwingungsdauern zur Bestimmung der Regression des B-Feldes. \cite{uncertainties}}
  \label{tab:MM2}
  \sisetup{table-format=2.2}
  \begin{tabular}{S[table-format=1.1] S @{${}\pm{}$} S[table-format=1.2] S @{${}\pm{}$} S[table-format=1.2] S[table-format=1.1] S[table-format=1.1]}
    \toprule
    {Index $i$} & \multicolumn{2}{c}{$\overline{T_\text{i}}/\si{\second}$} & \multicolumn{2}{c}{$\frac{1}{\overline{T_\text{i}}^2}/\frac{\mathrm{m}}{\si{\second\squared}}$} & {$I_\text{i}/\si{\ampere}$} & {$B/\si{\milli\tesla}$} \\
    \midrule
    0.1 & 17.92 & 0.00 & 3.11 & 0.00 & 0.1 & 0.1 \\
    0.2 & 15.47 & 0.02 & 4.18 & 0.01 & 0.2 & 0.2 \\
    0.3 & 15.18 & 2.40 & 4.34 & 1.37 & 0.3 & 0.3 \\
    0.4 & 12.95 & 0.02 & 5.96 & 0.02 & 0.4 & 0.4 \\
    0.5 & 11.76 & 0.01 & 7.23 & 0.02 & 0.5 & 0.5 \\
    0.6 & 11.02 & 0.01 & 8.23 & 0.02 & 0.6 & 0.6 \\
    0.7 & 10.36 & 0.01 & 9.32 & 0.02 & 0.7 & 0.7 \\
    0.8 & 10.03 & 0.01 & 9.94 & 0.02 & 0.8 & 0.8 \\
    0.9 & 9.35 & 0.02 & 11.43 & 0.06 & 0.9 & 0.9 \\
    1.0 & 9.00 & 0.00 & 12.35 & 0.01 & 1.0 & 1.0 \\
    \bottomrule
  \end{tabular}
\end{table}


\FloatBarrier
\subsection{Erdmagnetfeld}
\label{sec:Erdmagnetfeld2}
Aus den Schwingungsdauern $T$ ohne eingeschaltetem Magnetfeld und dem in Abschnitt
\ref{sec:MagnetischerMoment} berechneten magnetischen Moments $m = \SI{0.65(2)}{\ampere\meter\squared}$
lässt sich die horizontal Komponente des Erdmagnetfeldes mit der umgestellten Formel
\eqref{eqn:magPeriode} und der Formel \eqref{eqn:Gauß} auf
\begin{equation*}
  B_\text{h} = \SI{17.5(5)}{\micro\tesla}
\end{equation*}
bestimmen.

\begin{table}
  \centering
  \caption{Messwerte zur Bestimmung der horizontal Komponente des Erdmagnetfeldes. \cite{uncertainties}}
  \label{tab:Erdmagnetfeld}
  \sisetup{table-format=1.2}
  \begin{tabular}{S[table-format=2.3] S S @{${}\pm{}$} S}
    \toprule
    {$T/\si{\second}$} & {$\frac{1}{T^2}/\frac{\mathrm{m}}{\si{\second\squared}}$} & \multicolumn{2}{c}{$B/\si{\milli\tesla}$} \\
    \midrule
    17.940 & 3.11 & 0.02 & 0.0 \\
    17.935 & 3.11 & 0.02 & 0.0 \\
    17.934 & 3.11 & 0.02 & 0.0 \\
    17.925 & 3.11 & 0.02 & 0.0 \\
    17.926 & 3.11 & 0.02 & 0.0 \\
    17.920 & 3.11 & 0.02 & 0.0 \\
    17.915 & 3.12 & 0.02 & 0.0 \\
    17.911 & 3.12 & 0.02 & 0.0 \\
    17.906 & 3.12 & 0.02 & 0.0 \\
    17.900 & 3.12 & 0.02 & 0.0 \\
    17.898 & 3.12 & 0.02 & 0.0 \\
    17.896 & 3.12 & 0.02 & 0.0 \\
    \bottomrule
  \end{tabular}
\end{table}

