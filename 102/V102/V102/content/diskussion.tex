\section{Diskussion}
\label{sec:Diskussion}
Beim Vergleich zwischen dem im Abschnitt \ref{sec:Schubmodul} ermitteltem
Schubmodul $G = \SI{69.7(28)}{\giga\pascal}$ und dem Literaturwert
$G_\text{Lit} = \SI{81}{\giga\pascal}$ \cite{Stahl} ergibt sich eine Abweichung
von \SI{-14}{\percent}. Es kann ausgesagt werden, dass es sich bei dem untersuchten
Material um Stahl handelt, allerdings nicht um welche Sorte, da es eine
Vielzahl an Sorten mit gleichem Schubmodul \cite{Stahl} gibt.

Für das magnetische Moment $m = \SI{0.65(2)}{\ampere\meter\squared}$ in
Abschnitt \ref{sec:MagnetischerMoment} kann keine
Aussage über das tatsächliche magnetische Moment getroffen werden, da nicht bekannt ist,
welcher Magnet in der Kugel verbaut ist.

Die in Abschnitt \ref{sec:Erdmagnetfeld} berechnete horizontal Komponente des
Erdmagnetfeldes $B_\text{h} = \SI{17.5(5)}{\micro\tesla}$ weicht nur um \SI{-12.6}{\percent}
vom Literaurwert $B_{\text{h}_\text{Lit}} = \SI{20}{\micro\tesla}$ \cite{Biosensor} ab.

Die zuvor erwähnten Abweichungen können durch einige Gründe erklärt werden. Der
größte Fehler entstand vermutlich durch die Licht gesteuerte Messuhr, denn
durch Hintergrundstrahlung wurde die Messung beeinflusst. Ebenso wurde die Messung
zwischendurch unterbrochen, da sich Personen hinter dem Versuchaufbaus bewegten,
was zu einer vorzeitigten Auslösung der Messung führte. Weiterhin konnte die
Kugel nur nach Augenmaß und nur durch einen Punkt auf der Kugel möglichst waagerecht,
beziehungsweise in Süd-Nord-Richtung ausgerichtet werden. Da es nur einen Punkt
auf der Kugel gab, welcher die Position des Magneten kennzeichnete, konnte
somit die absulute Lage des Magneten nicht bestimmt werden.
Ein Pendeln der Kugel konnte trotz Verwendung der Bremse nicht vermieden werden.
Beim Versuchsteil aus Abschnitt \ref{sec:magMoment} war der Strom nur über die
Spannung zu regulieren und die Einteilung der Stromskala war nur grob, sodass
ein genaues Einstellen des Stroms nicht möglich war.
