\section{Diskussion}
\label{sec:Diskussion}
In allen Versuchsanteilen wurden die Magnetfelder verschiedener Spulen untersucht. Eine Zusammenfassung der Ergebnisse sowie die Abweichungen von den Theoriewerten befinden sich in der Tabelle \ref{tab:Ergebnisse}.

In der Abbildung \ref{fig:langespule} ist die konstante magnetische Flussdichte gut erkennbar. Wegen der Messaparatur konnten nicht bis zum Ende der Spule alle Messdaten notiert werden, deswegen ist das Abfallen der magnetischen Flussdichte nicht ersichtlich. Das Magnetfeld einer langen Spule ist innerhalb konstant, weil die Feldlinien parallel zur Spulenachse laufen. Außerhalb der Spule (zum Beispiel am Ende und am Anfang) ist jedoch das Magnetfeld inhomogen.

In der Abbildung \ref{fig:kurzespule} ist keine konstante magnetische Flussdichte erkennbar, da es um eine kurze Spule handelt. Weil die Länge der Spulen viel kleiner als der Spulendurchmesser ist, wird in der Abbildung eine Parabel visuallisiert, genauer ein Maximum am Punkt $ x = \SI{-0,12}{\meter}$. 

Für den nächsten Versuchsanteil ist bei den ersten Abstand aus der Abbildung \ref{fig:abstand1} innerhalb der Spule eine konstante magnetische Flussdichte und außerhalb einen linearen Abfall zu sehen. Bei den zweiten Abstand(s. Abbildung \ref{fig:abstand2}) ist jedoch einen leichten exponentiellen Abfall außerhalb der Spule und innerhalb noch homogen zu beobachten. Bei dem letzten(s. Abbildung \ref{fig:abstand3}) wird ein parabolisches Profil innerhalb der Spule und außerhalb einen deutlichen exponentiellen Abfall beobachtet.

Bei der Hysteresenkurve ist das H-Feld abhängig vom Strom, deshalb wird die magnetische Feldstärke $H$ gegen das Magnetfeld $B$ aufgetragen. Hier können keine Theoriewerten berechnet werden, wodurch die experimentellen Werte mit der Theorie nicht vergleichbar sind.

\subsection{Zusammenfassung}
\begin{table}[htbp]
	\centering
	\caption{Zusammenfassung der Ergebnisse.}
	\label{tab:Ergebnisse}
	\begin{tabular}{c c c c}
		\toprule
		Spulentyp & $B_\text{theo} / \SI{E-3}{\tesla}$ &  $B_\text{exp} / \SI{E-3}{\tesla}$ & Abweichung $ / \si{\percent}$\\
		\midrule
	    Lange Spule & 2,005 & 2,597 & 28,63 \\
	    Kurze Spule & 4,435 & 2,174 & 50,98 \\
	    \midrule 
	    Helmholtzspulenpaar
	     & & & \\
	    \midrule
	    Erster Abstand & 1,201 & 1,575 & 31,14 \\
	    Zweiter Abstand & 1,074 & 1,268 & 18,06 \\
	    Dritter Abstand & 8,506 & 3,320 & 60,96 \\
		\bottomrule
	\end{tabular}
\end{table}
