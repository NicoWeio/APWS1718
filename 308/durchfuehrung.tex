\section{Durchführung}
\label{sec:Durchführung}

\subsection{Magnetfeld einer Spule}
Eine lange Spule wird an ein Netzgerät angeschlossen. Durch einstellen von Strom und Spannung kann so ein Magnetfeld erzeugt werden. Gemessen wird dieses mit einer longitudinalen Hallsonde, die nach und nach in die
Spule geschoben wird. Dabei werden die Messwerte notiert. Im Anschluss wird dies mit einer kurzen Spule wiederholt.

\subsection{Magnetfeld eines Spulenpaares}
In diesem Teil wird das Magnetfeld eines Helmholtz-Spulenpaares untersucht. Dazu werden die Spulen mit einem Netzteil in Reihe geschaltet und Strom und Spannung werden in einem zulässigen Bereich ($< \SI{5}{\ampere}$) eingestellt.
Dann wird mit einer transversalen Sonde das Magnetfeld innerhalb und außerhalb des Spulenpaares gemessen und die Werte werden notiert. Wiederholt wird dies für zwei weitere unterschiedliche Spulenabstände.

\subsection{Hysteresekurve}
Eine Ringspule wird an ein Netzteil angeschlossen, sodass ein Magnetfeld aufgebaut werden kann. Das Feld innerhalb des Ringes wird mit einer transversalen Sonde gemessen. Nun wird schrittweise das Magnetfeld erhöht und 
dann wieder runtergeregelt. Anschließend wird der Strom umgepolt und das Verfahren wird für das Gegenfeld wiederholt. Nach nochmaligem Umpolen wird dann ein letztes Mal das Magnetfeld auf- und abgebaut.